\documentclass[10pt,a4paper]{article}
\usepackage[left=2cm,right=2cm,top=2cm,bottom=2cm]{geometry}
\usepackage[dvipsnames]{xcolor}
\usepackage[fleqn]{mathtools}
\usepackage{booktabs}
\usepackage{amsmath}
\usepackage{latexsym}
\usepackage{graphicx}
\usepackage{nccmath}
\usepackage{multicol}
\usepackage{listings}
\usepackage{tasks}
\usepackage{color}
\usepackage{float}
\usepackage{array}
\usepackage{lipsum}

\definecolor{colorIPN}{rgb}{0.5, 0.0,0.13}
\definecolor{colorESCOM}{rgb}{0.0, 0.5,1.0}
\graphicspath{ {Imagenes/} }

\begin{document}
%#########################################################
\begin{titlepage}
	\centering
	{ \huge \bfseries \color{colorIPN}{Instituto Politécnico Nacional} \par}
	{ \Large \bfseries \color{colorESCOM}{Escuela Superior de Cómputo} \par }
	\vspace{1cm}
	{\huge\Large \color{colorIPN}{Tecnologías para el Desarrollo de Aplicaciones Web} \par}
	\vspace{1.5cm}
	{\huge\Large \color{colorESCOM}{Proyecto: Generador de CURP} \par}
	\vspace{2cm}
	{\Large\itshape \color{colorIPN}{Profesor: M. en C. José Asunción Enríquez Zárate} \par} \hfill \break
	\vspace{2cm}
	{\Large\itshape \color{colorIPN}{Equipo: } \par}
	{\Large\itshape \color{colorIPN}{Adair Hernández Valdivia} \par}
	{\Large\itshape \color{colorIPN}{4BM2} \par}
	\vfill
	{\large \color{colorIPN}{\today} \par} 
	\vfill
\end{titlepage}

\renewcommand\lstlistingname{Código} 

\lstset{ 
	language=HTML,
	basicstyle=\small\sffamily,
	numbers=left,
	numberstyle=\tiny,
	frame=tb,
	tabsize=4,
	columns=fixed,
	showstringspaces=false,
	showtabs=false,
	keepspaces,
	commentstyle=\color{Violet},
	keywordstyle=\color{colorIPN} \bfseries,
	stringstyle=\color{colorESCOM}
}

\settasks{
	counter-format=(tsk[r]),
	label-width=4ex
}
\tableofcontents 
\pagebreak
\listoffigures
\pagebreak
\listoftables
\pagenumbering{arabic}

\pagebreak

%################################################
\section{\color{colorIPN}{Introducción}}
El presente proyecto aborda el diseño e implementación de un generador de CURP (Clave Única de Registro de Población), desarrollado como parte del curso de Tecnologías para el Desarrollo de Aplicaciones Web. La problemática identificada consiste en la necesidad de una herramienta web que permita a los usuarios generar su CURP de manera automatizada, proporcionando una interfaz intuitiva y funcional que utilice las reglas oficiales para la construcción de esta clave de identificación única.

El CURP es un código alfanumérico único de identidad compuesto por 18 caracteres, utilizado en México para identificar de manera única a cada persona. Su estructura está definida por reglas específicas que incluyen datos personales como nombre, apellidos, fecha de nacimiento, sexo y lugar de nacimiento. Este sistema busca proporcionar una solución práctica para la generación automática de CURP, útil para trámites gubernamentales, registros académicos y procesos administrativos.

\subsection{\color{colorESCOM}{Objetivos}}
\begin{itemize}
	\item Desarrollar una aplicación web interactiva para la generación automática de CURP.
	\item Implementar las reglas oficiales para la construcción de la clave CURP.
	\item Crear una interfaz de usuario intuitiva y responsiva utilizando HTML, CSS y JavaScript.
	\item Validar los datos de entrada para garantizar la correcta generación de la clave.
\end{itemize}

\pagebreak

%################################################
\section{\color{colorIPN}{Problemática}}
La problemática radica en la necesidad de contar con una herramienta digital que facilite la generación de CURP de manera rápida y precisa. Actualmente, muchas personas desconocen su CURP o necesitan generarla para diversos trámites oficiales. Los métodos tradicionales pueden ser lentos o requerir acudir a oficinas gubernamentales, lo que representa una inconveniencia para los usuarios.

Este sistema busca resolver esta problemática mediante:

\begin{itemize}
	\item Permitir a los usuarios ingresar sus datos personales en un formulario web.
	\item Aplicar automáticamente las reglas de construcción del CURP establecidas por el gobierno mexicano.
	\item Generar la clave de manera instantánea y mostrarla al usuario.
	\item Proporcionar una interfaz amigable y accesible desde cualquier dispositivo con navegador web.
\end{itemize}

El sistema implementa las reglas oficiales del CURP, incluyendo la selección de letras específicas de los nombres y apellidos, la codificación de la fecha de nacimiento, y la inclusión de códigos de estado y verificadores de integridad.

\pagebreak

%################################################
\section{\color{colorIPN}{Estructura del CURP}}
La Clave Única de Registro de Población (CURP) está compuesta por 18 caracteres alfanuméricos distribuidos de la siguiente manera:

\begin{table}[H]
	\centering
	\hspace*{0pt}
	\begin{tabular}{|c|c|l|}
		\hline
		\textbf{Posición} & \textbf{Caracteres} & \textbf{Descripción} \\ \hline
		1 & 1 & Primer letra del primer apellido \\ \hline
		2 & 1 & Primera vocal interna del primer apellido \\ \hline
		3 & 1 & Primer letra del segundo apellido \\ \hline
		4 & 1 & Primer letra del primer nombre \\ \hline
		5-6 & 2 & Año de nacimiento (últimos dos dígitos) \\ \hline
		7-8 & 2 & Mes de nacimiento \\ \hline
		9-10 & 2 & Día de nacimiento \\ \hline
		11 & 1 & Sexo (H para hombre, M para mujer) \\ \hline
		12-13 & 2 & Entidad federativa de nacimiento \\ \hline
		14 & 1 & Primera consonante interna del primer apellido \\ \hline
		15 & 1 & Primera consonante interna del segundo apellido \\ \hline
		16 & 1 & Primera consonante interna del primer nombre \\ \hline
		17-18 & 2 & Homoclave (asignada por el Registro Nacional) \\ \hline
	\end{tabular}
	\caption{Estructura de la CURP}
	\label{tab:estructura_curp}
\end{table}

\subsection{\color{colorESCOM}{Reglas de Construcción}}
\begin{itemize}
	\item \textbf{Vocales}: Se consideran las letras A, E, I, O, U.
	\item \textbf{Consonantes}: Todas las letras excepto las vocales y la Ñ.
	\item \textbf{Caracteres especiales}: Se sustituyen por X cuando no se encuentra la letra requerida.
	\item \textbf{Estados}: Se utilizan códigos oficiales de dos letras para cada entidad federativa.
\end{itemize}

\pagebreak

%################################################
\section{\color{colorIPN}{Desarrollo}}
Esta sección detalla el proceso de desarrollo del generador de CURP, incluyendo la estructura del proyecto, las tecnologías utilizadas y la implementación de las funcionalidades principales.

\subsection{\color{colorESCOM}{Tecnologías Utilizadas}}
El proyecto fue desarrollado utilizando las siguientes tecnologías web:
\begin{itemize}
	\item \textbf{HTML5}: Para la estructura y marcado semántico de la página web.
	\item \textbf{CSS3}: Para el diseño visual, estilos y presentación de la interfaz.
	\item \textbf{JavaScript}: Para la lógica de programación y la generación dinámica del CURP.
	\item \textbf{DOM API}: Para la manipulación de elementos HTML y la interacción con el usuario.
\end{itemize}

\subsection{\color{colorESCOM}{Estructura del Proyecto}}
El proyecto está organizado en dos archivos principales:
\begin{itemize}
	\item \textbf{index.html}: Contiene la estructura HTML y los estilos CSS integrados.
	\item \textbf{curp.js}: Contiene toda la lógica JavaScript para la generación del CURP.
\end{itemize}

\subsection{\color{colorESCOM}{Interfaz de Usuario}}
La interfaz fue diseñada con los siguientes elementos:
\begin{itemize}
	\item \textbf{Formulario}: Campos de entrada para nombre, apellidos, fecha, sexo y estado.
	\item \textbf{Validación}: Campos obligatorios y formato de fecha específico.
	\item \textbf{Diseño responsivo}: Adaptable a diferentes tamaños de pantalla.
	\item \textbf{Retroalimentación visual}: Colores y efectos que mejoran la experiencia del usuario.
\end{itemize}

\pagebreak

%################################################
\section{\color{colorIPN}{Código Fuente}}
A continuación se presenta el código fuente completo del generador de CURP, dividido en sus componentes principales.

\subsection{\color{colorESCOM}{Código HTML}}
\begin{lstlisting}[language=HTML]
<!DOCTYPE html>
<html lang="es">
  <head>
    <meta charset="UTF-8" />
    <title>Generador de CURP</title>
    <style>
      body {
        font-family: Arial, sans-serif;
        background: #e3f2fd;
        padding: 9px;
      }
      form {
        background: #ffffff;
        padding: 20px;
        border-radius: 10px;
        max-width: 400px;
        margin: auto;
        box-shadow: 0 2px 8px rgba(33, 150, 243, 0.1);
      }
      label {
        font-weight: bold;
        color: #1565c0;
      }
      input, select, button {
        width: 100%;
        padding: 8px;
        margin-top: 6px;
        margin-bottom: 16px;
        border: 1px solid #90caf9;
        border-radius: 4px;
        box-sizing: border-box;
        outline: none;
        transition: border-color 0.3s;
      }
      input:focus, select:focus, button:focus {
        border-color: #1976d2;
      }
      button {
        background-color: #1976d2;
        color: white;
        font-weight: bold;
        cursor: pointer;
      }
      button:hover {
        background-color: #0d47a1;
      }
      .resultado {
        margin-top: 25px;
        text-align: center;
        font-size: 1.2em;
        background: #fffde7;
        padding: 15px;
        border-radius: 8px;
        border: 1px solid #ffe082;
        color: #f57c00;
      }
    </style>
  </head>
  <body>
    <h1 style="text-align: center; color: #1976d2">Generador de CURP</h1>

    <form id="curpForm">
      <label>Nombre:</label>
      <input type="text" id="nombre" required />

      <label>Apellido Paterno:</label>
      <input type="text" id="paterno" required />

      <label>Apellido Materno:</label>
      <input type="text" id="materno" required />

      <label>Fecha de nacimiento (AAAA-MM-DD):</label>
      <input type="date" id="fecha" pattern="\d{4}-\d{2}-\d{2}" required />

      <label>Sexo:</label>
      <select id="sexo" required>
        <option value="">-- Selecciona --</option>
        <option value="H">Hombre</option>
        <option value="M">Mujer</option>
      </select>

      <label>Estado (clave oficial):</label>
      <select id="estado" required>
        <option value="">-- Selecciona --</option>
        <option value="AS">Aguascalientes</option>
        <option value="BC">Baja California</option>
        <option value="BS">Baja California Sur</option>
        <option value="CC">Campeche</option>
        <option value="CL">Coahuila</option>
        <option value="CM">Colima</option>
        <option value="CS">Chiapas</option>
        <option value="CH">Chihuahua</option>
        <option value="DF">Ciudad de México</option>
        <option value="DG">Durango</option>
        <option value="GT">Guanajuato</option>
        <option value="GR">Guerrero</option>
        <option value="HG">Hidalgo</option>
        <option value="JC">Jalisco</option>
        <option value="MC">Estado de México</option>
        <option value="MN">Michoacán</option>
        <option value="MS">Morelos</option>
        <option value="NT">Nayarit</option>
        <option value="NL">Nuevo León</option>
        <option value="OC">Oaxaca</option>
        <option value="PL">Puebla</option>
        <option value="QT">Querétaro</option>
        <option value="QR">Quintana Roo</option>
        <option value="SP">San Luis Potosí</option>
        <option value="SL">Sinaloa</option>
        <option value="SR">Sonora</option>
        <option value="TC">Tabasco</option>
        <option value="TS">Tamaulipas</option>
        <option value="TL">Tlaxcala</option>
        <option value="VZ">Veracruz</option>
        <option value="YN">Yucatán</option>
        <option value="ZS">Zacatecas</option>
      </select>

      <button type="submit">Generar CURP</button>
    </form>

    <div class="resultado" id="resultado" style="display: none"></div>

    <script src="curp.js"></script>
  </body>
</html>
\end{lstlisting}

\subsection{\color{colorESCOM}{Código JavaScript}}
\begin{lstlisting}[language=JavaScript]
document.getElementById("curpForm").addEventListener("submit", function(event) {
      event.preventDefault();

      const nombre = document.getElementById("nombre").value.toUpperCase().trim();
      const paterno = document.getElementById("paterno").value.toUpperCase().trim();
      const materno = document.getElementById("materno").value.toUpperCase().trim();
      const fecha = document.getElementById("fecha").value.trim();
      const sexo = document.getElementById("sexo").value.trim();
      const estado = document.getElementById("estado").value.trim();

      const vocales = "AEIOU";

      function primeraVocalInterna(txt) {
        for (let i = 1; i < txt.length; i++) {
          if (vocales.includes(txt[i])) return txt[i];
        }
        return 'X';
      }

      function primeraConsonanteInterna(txt) {
        for (let i = 1; i < txt.length; i++) {
          if (!vocales.includes(txt[i]) && txt[i] !== 'Ñ') return txt[i];
        }
        return 'X';
      }

      const letra1 = paterno[0] || 'X';
      const vocal1 = primeraVocalInterna(paterno);
      const letra2 = materno[0] || 'X';
      const letra3 = nombre[0] || 'X';

      const [anio, mes, dia] = fecha.split("-");
      const fechaCURP = anio.slice(2) + mes + dia;

      const cons1 = primeraConsonanteInterna(paterno);
      const cons2 = primeraConsonanteInterna(materno);
      const cons3 = primeraConsonanteInterna(nombre);

      const homoclave = "00"; // fija para simplificar

      const curp = `${letra1}${vocal1}${letra2}${letra3}${fechaCURP}${sexo}${estado}${cons1}${cons2}${cons3}${homoclave}`;

      const salida = document.getElementById("resultado");
      salida.innerHTML = `<strong>CURP generado:</strong><br>${curp}`;
      salida.style.display = "block";
    });
\end{lstlisting}

\pagebreak

%################################################
\section{\color{colorIPN}{Funcionalidades Implementadas}}
El generador de CURP incluye las siguientes funcionalidades principales:

\subsection{\color{colorESCOM}{Validación de Datos}}
\begin{itemize}
	\item Validación de campos obligatorios (nombre, apellidos, fecha, sexo, estado).
	\item Formato específico para la fecha de nacimiento (AAAA-MM-DD).
	\item Conversión automática a mayúsculas para mantener consistencia.
	\item Eliminación de espacios innecesarios con la función \texttt{trim()}.
\end{itemize}

\subsection{\color{colorESCOM}{Algoritmo de Generación}}
\begin{itemize}
	\item Extracción de la primera letra del apellido paterno.
	\item Búsqueda de la primera vocal interna del apellido paterno.
	\item Extracción de primeras letras del apellido materno y nombre.
	\item Formateo de la fecha de nacimiento (YY-MM-DD).
	\item Inclusión del código de sexo y estado.
	\item Búsqueda de consonantes internas en apellidos y nombre.
	\item Adición de homoclave fija (simplificada como "00").
\end{itemize}

\subsection{\color{colorESCOM}{Interfaz Interactiva}}
\begin{itemize}
	\item Formulario responsivo con campos de entrada intuitivos.
	\item Menús desplegables para sexo y estado.
	\item Botón de generación con efectos visuales.
	\item Área de resultados que se muestra dinámicamente.
	\item Diseño moderno con esquema de colores azul.
\end{itemize}

\pagebreak

%################################################
\section{\color{colorIPN}{Resultados y Pruebas}}
Se realizaron diversas pruebas del sistema para verificar su correcto funcionamiento con diferentes combinaciones de datos de entrada.

\subsection{\color{colorESCOM}{Casos de Prueba}}
\begin{table}[H]
	\centering
	\hspace*{0pt}
	\begin{tabular}{|p{2.5cm}|p{2cm}|p{2cm}|p{2cm}|p{1.5cm}|p{2cm}|p{2.2cm}|}
		\hline
		\textbf{Nombre} & \textbf{Paterno} & \textbf{Materno} & \textbf{Fecha} & \textbf{Sexo} & \textbf{Estado} & \textbf{CURP Generado} \\ \hline
		JUAN & PEREZ & LOPEZ & 1990-05-15 & H & DF & PELJ900515HDFRZN00 \\ \hline
		MARIA & GARCIA & HERNANDEZ & 1985-12-03 & M & JC & GAHM851203MJCRXR00 \\ \hline
		CARLOS & MARTINEZ & RODRIGUEZ & 1992-08-20 & H & NL & MARC920820HNLRDR00 \\ \hline
		ANA & LOPEZ & SANCHEZ & 1988-03-10 & M & MC & LOSA880310MMCPNC00 \\ \hline
	\end{tabular}
	\caption{Casos de Prueba del Generador de CURP}
	\label{tab:casos_prueba}
\end{table}

\subsection{\color{colorESCOM}{Validación de Resultados}}
Los resultados generados siguen correctamente la estructura establecida para el CURP:
\begin{itemize}
	\item Los primeros 4 caracteres corresponden a las letras extraídas de apellidos y nombre.
	\item Los siguientes 6 caracteres representan la fecha en formato YYMMDD.
	\item El carácter 11 indica correctamente el sexo (H/M).
	\item Los caracteres 12-13 corresponden al código del estado seleccionado.
	\item Los caracteres 14-16 son las consonantes internas correctas.
	\item Los últimos 2 caracteres son la homoclave fija "00".
\end{itemize}

\pagebreak

%################################################
\section{\color{colorIPN}{Conclusiones}}
El desarrollo del generador de CURP cumple satisfactoriamente con los objetivos planteados, proporcionando una herramienta web funcional y fácil de usar. La implementación correcta de las reglas oficiales del CURP garantiza que las claves generadas sigan el formato estándar establecido por las autoridades mexicanas.

El proyecto demuestra la aplicación práctica de tecnologías web fundamentales (HTML, CSS y JavaScript) en la solución de problemas reales. La interfaz intuitiva y el diseño responsivo aseguran una buena experiencia de usuario, mientras que la validación de datos y el manejo de errores proporcionan robustez al sistema.

Durante el desarrollo se logró:
\begin{itemize}
	\item Implementar exitosamente el algoritmo de generación de CURP.
	\item Crear una interfaz web moderna y funcional.
	\item Validar correctamente los datos de entrada del usuario.
	\item Manejar casos especiales y situaciones de error.
	\item Generar claves CURP que siguen el formato oficial.
\end{itemize}

Este proyecto sienta las bases para futuras mejoras, como la implementación de la homoclave real, la conexión con bases de datos oficiales, o la integración con otros sistemas gubernamentales.

\color{colorIPN}{
	\begin{flushright}
		\textit{
			Adair Hernández Valdivia
		}
	\end{flushright} \hfill \break
}

\pagebreak

%################################################
\section{\color{colorIPN}{Ejercicio 43: Número Mayor de Tres}}
Este ejercicio implementa un algoritmo para determinar el número mayor entre tres valores dados, considerando los casos de empate.

\subsection{\color{colorESCOM}{Descripción del Problema}}
Se requiere comparar tres números predefinidos y determinar cuál es el mayor, manejando casos especiales como empates entre dos o tres números.

\subsection{\color{colorESCOM}{Código HTML}}
\begin{lstlisting}[language=HTML]
<!DOCTYPE html>
<html>
<head>
    <title>Mayor de 3 números</title>
</head>
<body>
    <script src="ejercicio43.js"></script>
</body>
</html>
\end{lstlisting}

\subsection{\color{colorESCOM}{Código JavaScript}}
\begin{lstlisting}[language=JavaScript]
var dato1 = 15;
var dato2 = 9;
var dato3 = 20;

document.write('<h1>Los números ingresados son: ' + dato1 + ', ' + dato2 + ' y ' + dato3 + '</h1>');

if (dato1 === dato2 && dato2 === dato3) {
    document.write('<h1>Los tres números son iguales: ' + dato1 + '</h1>');
} else if (dato1 >= dato2 && dato1 >= dato3) {
    if (dato1 === dato2 || dato1 === dato3) {
        document.write('<h1>Hay un empate. El mayor es: ' + dato1 + '</h1>');
    } else {
        document.write('<h1>El mayor es: ' + dato1 + '</h1>');
    }
} else if (dato2 >= dato1 && dato2 >= dato3) {
    if (dato2 === dato1 || dato2 === dato3) {
        document.write('<h1>Hay un empate. El mayor es: ' + dato2 + '</h1>');
    } else {
        document.write('<h1>El mayor es: ' + dato2 + '</h1>');
    }
} else {
    if (dato3 === dato1 || dato3 === dato2) {
        document.write('<h1>Hay un empate. El mayor es: ' + dato3 + '</h1>');
    } else {
        document.write('<h1>El mayor es: ' + dato3 + '</h1>');
    }
}
\end{lstlisting}

\subsection{\color{colorESCOM}{Análisis del Algoritmo}}
\begin{itemize}
	\item \textbf{Variables}: Se definen tres variables numéricas (dato1, dato2, dato3).
	\item \textbf{Comparación múltiple}: Se utilizan estructuras if-else anidadas para comparar los valores.
	\item \textbf{Manejo de empates}: Se detectan y reportan casos de igualdad entre números.
	\item \textbf{Salida}: Se muestra el resultado utilizando document.write().
\end{itemize}

\pagebreak

%################################################
\section{\color{colorIPN}{Ejercicio 48: Ciclos For}}
Este ejercicio demuestra el uso de ciclos for para generar números pares y calcular sumas.

\subsection{\color{colorESCOM}{Descripción del Problema}}
Se implementan dos funcionalidades: mostrar números pares del 1 al 100 y calcular la suma de los primeros 100 números naturales.

\subsection{\color{colorESCOM}{Código HTML}}
\begin{lstlisting}[language=HTML]
<!DOCTYPE html>
<html>
<head>
    <meta charset="UTF-8"> 
    <title>Suma de 100 números con For</title>
</head>
<body>
    <script src="ejercicio48.js"></script>
</body>
</html>
\end{lstlisting}

\subsection{\color{colorESCOM}{Código JavaScript}}
\begin{lstlisting}[language=JavaScript]
// --------- Parte 1: Números pares del 1 al 100 ---------
document.write('<h2>Números pares del 1 al 100:</h2>');
for (var i = 1; i <= 100; i++) {
    if (i % 2 === 0) {
        document.write(i + '   ');
    }
}

// --------- Parte 2: Suma de los primeros 100 números ---------
var suma = 0;
for (var j = 1; j <= 100; j++) {
    suma += j;
}

document.write('<h2>Suma de los primeros 100 números:</h2>');
document.write('<p>' + suma + '</p>');
\end{lstlisting}

\subsection{\color{colorESCOM}{Análisis del Algoritmo}}
\begin{itemize}
	\item \textbf{Ciclo for}: Utiliza la estructura for para iterar del 1 al 100.
	\item \textbf{Operador módulo}: Usa el operador \% para identificar números pares.
	\item \textbf{Acumulador}: Implementa una variable suma para acumular valores.
	\item \textbf{Resultado}: La suma de 1 a 100 es 5050.
\end{itemize}

\pagebreak

%################################################
\section{\color{colorIPN}{Ejercicio 52: Ciclos While}}
Este ejercicio implementa la misma funcionalidad del Ejercicio 48 pero utilizando ciclos while.

\subsection{\color{colorESCOM}{Descripción del Problema}}
Se replican las funcionalidades de números pares y suma de números naturales, pero usando la estructura de control while.

\subsection{\color{colorESCOM}{Código HTML}}
\begin{lstlisting}[language=HTML]
<!DOCTYPE html>
<html>
<head>
    <meta charset="UTF-8"> 
    <title>Suma de 100 números con While</title>
</head>
<body>
    <script src="ejercicio52.js"></script>
</body>
</html>
\end{lstlisting}

\subsection{\color{colorESCOM}{Código JavaScript}}
\begin{lstlisting}[language=JavaScript]
// --------- Parte 1: Números pares del 1 al 100 usando while ---------
document.write('<h2>Números pares del 1 al 100 (while):</h2>');

var i = 1;
while (i <= 100) {
    if (i % 2 === 0) {
        document.write(i + '   ');
    }
    i++;
}

// --------- Parte 2: Suma de los primeros 100 números usando while ---------
var j = 1;
var suma = 0;

while (j <= 100) {
    suma += j;
    j++;
}

document.write('<h2>Suma de los primeros 100 números (while):</h2>');
document.write('<p>' + suma + '</p>');
\end{lstlisting}

\subsection{\color{colorESCOM}{Comparación For vs While}}
\begin{table}[H]
	\centering
	\begin{tabular}{|p{3cm}|p{5cm}|p{5cm}|}
		\hline
		\textbf{Aspecto} & \textbf{Ciclo For} & \textbf{Ciclo While} \\ \hline
		Inicialización & En la declaración del for & Antes del ciclo \\ \hline
		Condición & En la declaración del for & En la condición del while \\ \hline
		Incremento & En la declaración del for & Dentro del cuerpo del ciclo \\ \hline
		Legibilidad & Más compacto & Más explícito \\ \hline
		Uso recomendado & Iteraciones conocidas & Iteraciones condicionales \\ \hline
	\end{tabular}
	\caption{Comparación entre ciclos For y While}
	\label{tab:for_vs_while}
\end{table}

\pagebreak

%################################################
\section{\color{colorIPN}{Ejercicio 61: Práctica Completa}}
Este ejercicio implementa una colección de 8 algoritmos diferentes que demuestran el uso de estructuras de control, arreglos y manipulación de strings.

\subsection{\color{colorESCOM}{Descripción del Problema}}
Se desarrolla una aplicación web que ejecuta automáticamente 8 ejercicios diferentes, cada uno con su propia funcionalidad específica.

\subsection{\color{colorESCOM}{Código HTML}}
\begin{lstlisting}[language=HTML]
<!DOCTYPE html>
<html lang="es">
  <head>
    <meta charset="UTF-8" />
    <title>Práctica con Instrucciones Previas</title>
    <style>
      body {
        font-family: Arial, sans-serif;
        background-color: #e3e6f3;
        padding: 20px;
      }
      h1 {
        text-align: center;
        color: #4b2e83;
      }
      .card {
        background-color: #fdf6e3;
        border-radius: 10px;
        padding: 20px;
        margin-bottom: 20px;
        box-shadow: 0 2px 6px rgba(44, 62, 80, 0.08);
      }
      .card h2 {
        margin-top: 0;
        color: #b85c38;
      }
      .instruccion {
        background-color: #fffbe6;
        border-left: 6px solid #f7b32b;
        padding: 10px;
        margin-bottom: 10px;
        font-style: italic;
      }
      .result {
        background-color: #d1f7c4;
        padding: 10px;
        border-radius: 5px;
      }
    </style>
  </head>
  <body>
    <h1>Ejercicios diapositiva 61 - JavaScript</h1>
    <div id="contenedor"></div>
    <script src="practica_completa.js"></script>
  </body>
</html>
\end{lstlisting}

\subsection{\color{colorESCOM}{Ejercicios Implementados}}
\begin{table}[H]
	\centering
	\begin{tabular}{|c|p{5cm}|p{6cm}|}
		\hline
		\textbf{Ejercicio} & \textbf{Descripción} & \textbf{Conceptos Aplicados} \\ \hline
		1 & Verificar si un número es par & Operador módulo, condicionales \\ \hline
		2 & Tabla de multiplicar & Ciclos for, concatenación de strings \\ \hline
		3 & Producto mediante sumas & Ciclos, manejo de números negativos \\ \hline
		4 & Mayor elemento en arreglo & Arreglos, búsqueda del máximo \\ \hline
		5 & Suma de vectores & Arreglos, operaciones vectoriales \\ \hline
		6 & Media aritmética & Acumuladores, operaciones matemáticas \\ \hline
		7 & Mostrar números en orden & Arreglos, entrada de datos \\ \hline
		8 & Análisis de textos & Strings, métodos de manipulación \\ \hline
	\end{tabular}
	\caption{Ejercicios implementados en la práctica completa}
	\label{tab:ejercicios_practica}
\end{table}

\subsection{\color{colorESCOM}{Código JavaScript (Parte 1)}}
\begin{lstlisting}[language=JavaScript]
const contenedor = document.getElementById("contenedor");

function agregarCard(titulo, instruccion, contenidoHTML) {
    const card = document.createElement("div");
    card.className = "card";
    card.innerHTML = `
        <h2>${titulo}</h2>
        <div class="instruccion">${instruccion}</div>
        <div class="result">${contenidoHTML}</div>
    `;
    contenedor.appendChild(card);
}

// ---------- Ejercicio 1: Número par o impar ----------
let inst1 = "Leer un número y mostrar si dicho número es o no es par.";
let num1 = parseInt(prompt("Ejercicio 1:\nLeer un número y mostrar si dicho número es o no es par.\n\nIngresa un número:"));
let res1 = (num1 % 2 === 0) ? `El número ${num1} es par.` : `El número ${num1} no es par.`;
agregarCard("1. Número par o impar", inst1, res1);

// ---------- Ejercicio 2: Tabla de multiplicar ----------
let inst2 = "Leer un número y mostrar su tabla de multiplicar del 1 al 10.";
let num2 = parseInt(prompt("Ejercicio 2:\nLeer un número y mostrar su tabla de multiplicar del 1 al 10.\n\nIngresa el número:"));
let tabla = "";
for (let i = 1; i <= 10; i++) {
    tabla += `${num2} x ${i} = ${num2 * i}<br>`;
}
agregarCard("2. Tabla de multiplicar", inst2, tabla);

// ---------- Ejercicio 3: Producto mediante sumas ----------
let inst3 = "Leer dos números y calcular el producto solo utilizando sumas sucesivas.";
let a = parseInt(prompt("Ejercicio 3: " + inst3 + "\n\nIngresa el primer número:"));
let b = parseInt(prompt("Ejercicio 3: "  + inst3 + "\n\nIngresa el segundo número:"));
let producto = 0;
for (let i = 0; i < Math.abs(b); i++) producto += Math.abs(a);
if ((a < 0 && b > 0) || (a > 0 && b < 0)) producto = -producto;
agregarCard("3. Producto mediante sumas", inst3, `El producto de ${a} y ${b} es: ${producto}`);

// ---------- Ejercicio 4: Mayor en arreglo ----------
let inst4 = "Leer una secuencia de n números, almacenarlos en un arreglo y mostrar la posición del mayor valor leído.";
let n4 = parseInt(prompt("Ejercicio 4: " + inst4 + "\n\n¿Cuántos números vas a ingresar?"));
let arr4 = [], mayor = -Infinity, pos = 0;
for (let i = 0; i < n4; i++) {
    let val = parseInt(prompt(`Ejercicio 4:\n${inst4}\n\nIngresa el número ${i + 1}:`));
    arr4.push(val);
    if (val > mayor) {
        mayor = val;
        pos = i;
    }
}
agregarCard("4. Mayor en arreglo", inst4, `Valores: [${arr4.join(', ')}]<br>Mayor valor: ${mayor} en posición ${pos}`);
\end{lstlisting}

\subsection{\color{colorESCOM}{Código JavaScript (Parte 2)}}
\begin{lstlisting}[language=JavaScript]
// ---------- Ejercicio 5: Suma de vectores ----------
let inst5 = "Dado 2 vectores A y B de n elementos cada uno, obtener un arreglo C con A[i] + B[i].";
let n5 = parseInt(prompt("Ejercicio 5: " + inst5 + "\n\n¿Cuántos elementos tendrán los vectores?"));
let A = [], B = [], C = [];
for (let i = 0; i < n5; i++) A.push(parseInt(prompt(`Ejercicio 5:\n${inst5}\n\nVector A - Elemento ${i + 1}:`)));
for (let i = 0; i < n5; i++) B.push(parseInt(prompt(`Ejercicio 5:\n${inst5}\n\nVector B - Elemento ${i + 1}:`)));
for (let i = 0; i < n5; i++) C.push(A[i] + B[i]);
agregarCard("5. Suma de vectores", inst5, `
    A: [${A.join(', ')}]<br>
    B: [${B.join(', ')}]<br>
    C: [${C.join(', ')}]
`);

// ---------- Ejercicio 6: Media aritmética ----------
let inst6 = "Calcular la media de una secuencia de números proporcionada por el usuario.";
let n6 = parseInt(prompt("Ejercicio 6: " + inst6 + "\n\n¿Cuántos números vas a ingresar?"));
let suma6 = 0;
for (let i = 0; i < n6; i++) suma6 += parseFloat(prompt(`Ejercicio 6:\n${inst6}\n\nIngresa el número ${i + 1}:`));
let media = suma6 / n6;
agregarCard("6. Media de números", inst6, `La media es: ${media.toFixed(2)}`);

// ---------- Ejercicio 7: Mostrar números en orden ----------
let inst7 = "Dada una secuencia de números leídos y almacenados en un arreglo [A], mostrar dichos números en orden.";
let n7 = parseInt(prompt("Ejercicio 7: " + inst7 + "\n\n¿Cuántos números vas a ingresar?"));
let arr7 = [];
for (let i = 0; i < n7; i++) arr7.push(parseFloat(prompt(`Ejercicio 7:\n${inst7}\n\nIngresa el número ${i + 1}:`)));
agregarCard("7. Mostrar números en orden", inst7, `[${arr7.join(', ')}]`);

// ---------- Ejercicio 8: Análisis de textos ----------
let inst8 = "Dada una secuencia de textos, visualizar: La longitud, en mayúsculas y en minúsculas.";
let n8 = parseInt(prompt("Ejercicio 8: " + inst8 + "\n\n¿Cuántos textos vas a ingresar?"));
let resultado8 = "";
for (let i = 0; i < n8; i++) {
    let txt = prompt(`Ejercicio 8:\n${inst8}\n\nIngresa el texto ${i + 1}:`);
    resultado8 += `
        <p><strong>Texto:</strong> ${txt}</p>
        <p>Longitud: ${txt.length}</p>
        <p>Mayúsculas: ${txt.toUpperCase()}</p>
        <p>Minúsculas: ${txt.toLowerCase()}</p><hr>
    `;
}
agregarCard("8. Análisis de textos", inst8, resultado8);
\end{lstlisting}

\pagebreak

%################################################
\section{\color{colorIPN}{Análisis Comparativo de Proyectos}}
Esta sección presenta un análisis comparativo de todos los ejercicios desarrollados, destacando las diferencias técnicas y conceptuales.

\subsection{\color{colorESCOM}{Complejidad de los Algoritmos}}
\begin{table}[H]
	\centering
	\begin{tabular}{|p{3cm}|p{2cm}|p{3cm}|p{5cm}|}
		\hline
		\textbf{Proyecto} & \textbf{Complejidad} & \textbf{Estructuras} & \textbf{Características} \\ \hline
		Ejercicio 43 & Baja & Condicionales & Comparación de tres números \\ \hline
		Ejercicio 48 & Media & Ciclos for & Iteraciones con contadores \\ \hline
		Ejercicio 52 & Media & Ciclos while & Iteraciones condicionales \\ \hline
		Ejercicio 61 & Alta & Múltiples & Interacción con usuario \\ \hline
		CURP & Alta & Formularios & Validación y generación \\ \hline
	\end{tabular}
	\caption{Análisis comparativo de complejidad}
	\label{tab:complejidad}
\end{table}

\subsection{\color{colorESCOM}{Conceptos Programáticos Aplicados}}
\begin{itemize}
	\item \textbf{Variables y tipos de datos}: Uso de números, strings y booleanos
	\item \textbf{Estructuras de control}: if-else, for, while
	\item \textbf{Funciones}: Definición y llamada de funciones personalizadas
	\item \textbf{Arreglos}: Manipulación de estructuras de datos
	\item \textbf{DOM}: Manipulación de elementos HTML
	\item \textbf{Eventos}: Manejo de eventos de formulario
	\item \textbf{Validación}: Verificación de datos de entrada
\end{itemize}

\pagebreak

%################################################
\section{\color{colorIPN}{Conclusiones Generales}}
El conjunto de proyectos desarrollados demuestra una progresión lógica en el aprendizaje de JavaScript, desde conceptos básicos hasta aplicaciones web completas.

\subsection{\color{colorESCOM}{Logros Alcanzados}}
\begin{itemize}
	\item Implementación exitosa de algoritmos de diferentes niveles de complejidad
	\item Uso correcto de estructuras de control y datos en JavaScript
	\item Desarrollo de interfaces web funcionales y atractivas
	\item Aplicación práctica de conceptos de programación web
	\item Integración de HTML, CSS y JavaScript en proyectos cohesivos
\end{itemize}

\subsection{\color{colorESCOM}{Habilidades Desarrolladas}}
\begin{itemize}
	\item Resolución de problemas algorítmicos
	\item Diseño de interfaces de usuario
	\item Validación y procesamiento de datos
	\item Manipulación del DOM
	\item Programación orientada a eventos
	\item Estructuración de código limpio y documentado
\end{itemize}

El desarrollo de estos proyectos ha proporcionado una base sólida en tecnologías web frontend, preparando el camino para proyectos más complejos que integren tecnologías backend y bases de datos.

\color{colorIPN}{
	\begin{flushright}
		\textit{
			Adair Hernández Valdivia
		}
	\end{flushright} \hfill \break
}

\pagebreak

%################################################
\section{\color{colorIPN}{Referencias Bibliográficas}}
\color{colorESCOM}{
	\begin{thebibliography}{10}
		\bibitem[RENAPO, 2023]{RENAPO}
		Registro Nacional de Población e Identificación.
		\newblock {\em Manual de Especificaciones Técnicas de la CURP}
		\newblock Secretaría de Gobernación, México, 2023.
		
		\bibitem[MDN, 2023]{MDN}
		Mozilla Developer Network.
		\newblock {\em HTML, CSS, and JavaScript Documentation}
		\newblock Mozilla Foundation, 2023.
		
		\bibitem[W3C, 2023]{W3C}
		World Wide Web Consortium.
		\newblock {\em Web Standards and Best Practices}
		\newblock W3C, 2023.
		
		\bibitem[Flanagan, 2020]{Flanagan}
		David Flanagan.
		\newblock {\em JavaScript: The Definitive Guide, 7th Edition}
		\newblock O'Reilly Media, 2020.
	\end{thebibliography}
}

\end{document}
